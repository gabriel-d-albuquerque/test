\section{Geral}
\label{sec: texto}

Boa documentação para \LaTeX \href{https://en.wikibooks.org/wiki/LaTeX}{\textit{ aqui}}.

\subsection{Texto}

\textbf{Negrito}.

\textit{Itálico}.

\underline{Sublinhado}.

\textsc{Small caps}.

\begin{bfseries}
Série de texto em negrito
\end{bfseries}

\hl{Texto grifado}

\begin{center}
Este texto está centralizado.
\end{center}

\subsubsection{Tamanho da Fonte}

\tiny
Muito pequeno

\scriptsize
Menos pequeno

\footnotesize
Rodapé

\small
Pequeno

\normalsize
Normal

\large
Maior

\Large
Grande

\LARGE
Muito grande

\huge
Extremamente Grande

\Huge
Enorme

\normalsize


\subsection{Espaçamentos}

\hspace{1cm}
\vspace{1cm}

\section{Listas}

\begin{itemize}
    \item Item normal
    \item [$\star$] Esse é um item com estrela.
    \item [a] Item com a
    \item [$\dagger$] Esse é um item com cruz.
\end{itemize}

\vspace{1cm}

\begin{enumerate}
    \item Esse é o primeiro nível.
    \begin{enumerate}
        \item Esse é o segundo.
    \end{enumerate}
\end{enumerate}

\section{Figuras e tabelas}
\label{sec: fig/tab}

\subsection{Figuras}

\begin{figure}[H]
    \centering
    \includegraphics[width=15cm]{figures/sample.png}
    \caption{Legenda}
    \label{fig: sample}
\end{figure}

\subsection{Tabelas}

Acesse esse \href{https://www.tablesgenerator.com}{\textit{link}} para fazer sua tabela.

\section{Matemática}

Seja $f$ a função $f(x)=x^2$.
Então $f(2)=4$ e \[ f(-3)=9. \]

\begin{equation}
    x = \frac{-b \pm \sqrt{b^2 - 4ac}}{2a}
\label{eq: bhaskara}
\end{equation}

\[ \alpha \delta \Delta \psi \]

\[ c_{l\alpha} \]

\[ \int_2^{15} x^2\,dx \]

\[ \left(\frac{1}{5}\right\}^6 \]

\[ \lim_{x \to 1} f(x) \]

\[ \sum_{i=1}^n i \]

\begin{align}
a^2 + b^2 &= c^2 \\
a+b &= c + 2
\end{align}

\begin{equation}
\begin{Bmatrix}
a & b\\
c & d
\end{Bmatrix}
\end{equation}

\begin{theorem}[Teorema de Pitágoras]
Em um triângulo retângulo, o quadrado da hipotenusa é igual à soma dos quadrados dos catetos.
\end{theorem}

\section{Referências Cruzadas}

Segundo a Equação \ref{eq: bhaskara}

Na Figura \ref{fig: sample}

Na Seção \ref{sec: texto}

\section{Citação}

Segundo Houghton \cite{houghton}

\begin{table}[H]
\centering
\caption{Valores calculados de módulo de impulso sofrido (I), energia cinética e quantidade de movimento (P).}
\label{tab:4}
\resizebox{\textwidth}{!}{%
\begin{tabular}{|c|c|}
\hline
Parâmetro    & Valor                \\ \hline
$\Delta t_1$ & $(0,094 \pm 0,001)s$ \\ \hline
$\Delta t_2$ & $(0,099 \pm 0,001)s$ \\ \hline
\end{tabular}%
}
\end{table}